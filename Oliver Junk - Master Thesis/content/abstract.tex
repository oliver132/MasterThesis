\begin{abstractDE}
In den letzten Jahren hat der Gesundheitssektor Fortschritte bei der Digitalisierung seiner Prozesse und Ressourcen erzielt. Mit den Fortschritten in der Kommunikationstechnologie kamen allerdings auch ihre Nachteile: Elektronische Informationssysteme bieten Angriffsfläche für bösartige Akteure. Die Europäische Union hat die Datenschutz-Grundverordnung (DSGVO) in Kraft gesetzt, um die Privatsphäre ihrer Bürger zu schützen. In dieser Masterarbeit wird ein Sicherheitskonzept zur Sicherung der Weitergabe personenbezogener Daten und zur Einhaltung der Datenschutz-\\Grundverordnung für den Einsatz im CliniScale Projekt entworfen. Die Forschung beginnt mit einer systematischen Literaturrecherche und einer Analyse der DSGVO, um gesetzliche Anforderungen zu identifizieren, denen das in dieser Masterarbeit entwickelte Konzept entsprechen muss. Nach der Modellierung des CliniScale Systems wird eine Risikoanalyse zur Bewertung des Sicherheitsrisikos mithilfe der Modular Risk Assessment (MoRA) Methodik durchgeführt, um kritische Infrastrukturen zu identifizieren und Sicherheitsmaßnahmen zu implementieren, die in Betracht der aufgetretenen Risiken angemessen sind. Dabei wird die Sicherheit des erstellten Konzepts validiert. Für die identifizierten Sicherheitsmaßnahmen wird eine Implementierungsrichtlinie erstellt, die es ermöglicht, das Sicherheitskonzept nicht nur in der Umgebung des CliniScale Projekts, sondern für jede ähnliche Infrastruktur im Gesundheitswesen einzusetzen.
\end{abstractDE}

\vfill

\begin{abstractEN}
In recent years, the healthcare sector made advances in digitalizing their processes and resources. With advances in communication technology, also came their drawbacks: Electronic information is pruned to being exploited by adversaries. The European Union implemented the General Data Protection Regulation (GDPR) in order to protect the privacy of their citizen. In this thesis, a security concept securing the transfer of personal data and complying with the GDPR is designed for the use in the CliniScale project. The research starts with a systematic literature research and an analysis of the GDPR in order to identify regulations this thesis has to comply with. After modeling the CliniScale system environment, a security risk assessment is performed using the Modular Risk Assessment (MoRA) methodology to identify critical infrastructure and implement security measurements appropriate to the encountered risks. In the process, the security of the created concept is validated. An implementation guideline is created for the identified security measurements, enabling the implementation of the security concept not only in the environment of the CliniScale project, but for every similar infrastructure in the healthcare domain.
\end{abstractEN}

\vfill