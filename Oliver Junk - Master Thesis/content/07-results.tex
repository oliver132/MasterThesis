\chapter{Results}
\label{results}
The chapter \textit{\nameref{results}} consists of the sections \textit{\nameref{summary}} and \textit{\nameref{limitations}}. The first section, \textit{\nameref{summary}}, presents the findings of this thesis. In the second section, \textit{\nameref{limitations}}, aspects that are not part of this thesis are presented.

\section{Summary}
\label{summary}
In the context of this thesis a security risk assessment is performed and in the process a security concept for the secure transfer of sensitive personal data is designed and evaluated. \\
\newline
The MoRA methodology is modified to use the Microsoft STRIDE as threat catalogue and the controls defined in the Microsoft Threat Modeling Tool as control catalogue. Also, three activities of the MoRA methodology are altered in order to support the import of threats and controls generated by the Microsoft Threat Modeling Tool: \textit{Determine Protection Needs}, \textit{Analyze Threats} and \textit{Establish Controls}. As the MoRA methodology is built in a modular way, consisting of various activities and work products, alteration of single activities is possible if the consumed and produced artifact are the same.\\
\newline
The statutory regulations implemented by the GDPR are analyzed. The articles and recitals with legal consequences regarding this thesis are identified and statutory requirements to the security concept designed in this thesis are extracted and defined as requirements. Further, it is analyzed which requirements are covered in the process of performing a security risk assessment.\\
Next, the process of the CliniScale system is analyzed and modeled. The three parties involved in the process, the \textit{CliniScale Environment}, the \textit{Trial Executor} and the user of the \textit{CliniScale Application} for mobile devices, are identified. The process is further analyzed to create an abstract workflow used to model the system architecture for the security risk assessment at a later step.\\
Last, a reliable source for up-to-date cryptographic recommendations is identified in the BSI technical guidelines TR-02102-1 and TR-02102-2.\\
\newline
To perform a security risk assessment, preliminary tasks must be performed.\\
First, the \textit{\nameref{moraassmodel}} is configured to fit the needs of a security risk assessment investigating the CliniScale project. In the process, six security goal classes, three levels of damage potentials, two damage criteria partitioned in three damage subclasses, four levels of required attack potentials and three risk levels are defined.\\
\newline
The next step in preparation for the risk assessment is to model the system architecture as \textit{\nameref{morasud}}. Following the definition of the CliniScale process analyzed in \textit{\nameref{sysrequirements}}, functions, components, data elements and channels are defined according to the \textit{\nameref{moradocsud}} activity and the \textit{System Under Development (SUD)} is created.\\
To integrate the Microsoft tools into the MoRA methodology, the \textit{SUD} is modeled in the Microsoft Threat Modeling Tool.\\
Threat and control catalogues are synthesized of the treats and controls generated in the Microsoft Threat Modeling Tool. A set of the generated threats is categorized using the STRIDE threat model and imported into the risk assessment as threat catalogue. The threat catalogue created consists of 44 threat classes. Similar to threats, controls are categorized using the ten categories defined by the Microsoft Threat Modeling Tool and imported into the risk assessment as control catalogue. The control catalogue consists of 82 control classes.\\
With the creation of the artifacts \textit{\nameref{morathreatcat}} and \textit{\nameref{moracontrolcat}} and the work product \textit{\nameref{morasud}}, activities leading to the work product \textit{\nameref{moraassessment}} can be performed.\\
\newline
In order to perform the security risk assessment and create the work product \textit{\nameref{moraassessment}}, the four activities \textit{\nameref{altereddetprotneeds}}, \textit{\nameref{alteredanathreats}}, \textit{\nameref{moraanarisks}} and \textit{\nameref{alteredestacontrols}} are performed.\\
In the process of performing \textit{\nameref{altereddetprotneeds}} 54 security goals are defined, 24 security goals concerning data elements and 30 security goals concerning data flow elements. Of these security goals, 27 have the damage potential "High", 15 the damage potential "Moderate" and twelve the damage potential "Low". \\
\textit{\nameref{alteredanathreats}} is the next activity defined in the MoRA methodology. In the process, 38 threats split in the six STRIDE categories are defined.\\
The next activity is \textit{\nameref{moraanarisks}} to define of risk elements. In the process, 38 risk elements are defined. Of these, 36 cause a risk level of "High" and two cause a risk level of "Moderate". Since the average risk level is too high to be acceptable, controls mitigating the threats must be identified.\\
To mitigate the high risk levels, the activity \textit{\nameref{alteredestacontrols}} is performed. Split on ten categories, 72 controls are defined. The mitigating effect of these controls is notable in the risk elements: 15 risk elements now have a risk level of "Moderate" and 23 risks a risk level of "Low". Discussing the resulting risk levels and the implementation of the controls, it is determined that there is no need for a second iteration of the risk assessment as the risk levels are acceptable.\\
\newline
At last, an implementation guideline of the identified controls for the CliniScale system is created. The current implementation of the CliniScale system is analyzed by identifying the system architecture and the technologies used.\\
Categorized by the ten categories defined in the Threat Modeling Tool, for all 72 control classes practical implementation suggestions, if possible, using the already used frameworks and libraries of the current implementation of the CliniScale system, are given.

\section{Limitations}
\label{limitations}
The scope of this thesis being the design of a security concept for the transfer of sensitive data, it leaves open the question of how to store them locally in a secure manner. Secure storage, depending on the underlying system, is as important as securing the data transfer. Especially the storage of sensitive personal data and secret keys used for encryption possess a critical value.\\
\newline
While articles of the General Data Protection Regulation regarding the transfer of sensitive data have been identified, many more articles would apply to an information system in the health domain. It is subject of further research to identify these articles and their requirements. Issues, such as the requirement of a data protection impact assessment for article 35\cite{GDPR35} or the implementation of a user's right to access, erasure or restriction defined in the articles 15\cite{GDPR15}, 17\cite{GDPR17} and 18\cite{GDPR18}, have to be identified.\\
\newline
The Microsoft STRIDE threat model and the mitigation categories of the Microsoft Threat Modeling Tool offer a good overall coverage, being catalogues that are universally applicable. Although this implies that they may lack domain specific knowledge. Threats that are specific to the domain of information systems in healthcare may not be covered by the used catalogues and therefore the resulting risk level could be false. Another limitation of using catalogues based on the STRIDE model is the missing property of a threat's probability. Instead of estimating the required attack potential, the quality of the catalogues could have been improved by applying a risk rating model.\\
\newline
Configuring the \textit{Assessment Model} as a preliminary step for the security risk assessment is a task based on the operator’s experience. Some details, like defining security goal classes, are predefined based on the usage of the Microsoft STRIDE threat model. Definition of the damage classes and its subclasses lies in the experience of the operator. The MoRA methodology doesn't give a clear artifact on which classes to use, as it can vary from one domain to another. It is assumed that, besides the risk of a loss of self-determination and financial or reputational damage, there is no further damage class which warrants to be considered.\\


