%%%%%%%%%%%%% Packages %%%%%%%%%%%%%
\usepackage{beamerthemesplit}
\usepackage[ngerman,english]{babel}
%\usepackage{lmodern} % verbessert verwendete Schriftart
%\usepackage{times}
\usepackage{amsmath}
\usepackage{listings}
\usepackage{calc}
\usepackage[labelfont={color=DBbluedark,bf}, font={color=DBbluedark,footnotesize}]{caption}
\usepackage{geometry}
\usepackage{fontspec}
\usepackage{enumitem}
\usepackage{multicol}
\usepackage[draft]{pdfcomment}


%%%%%%%%%%%%% Settings %%%%%%%%%%%%%
\beamertemplatenavigationsymbolsempty% suppress the navigation bar
\setbeamertemplate{caption}[numbered] % Caption haben eine Nummer

%\beamertemplategridbackground[\abstand] % hinterlegt ein Gitter

\rowcolors{2}{lightgray!20}{lightgray!40} % alternierende Zeilenfarbe
\renewcommand{\arraystretch}{1.2}

\newcommand{\comment}[1]{}


%%%%%%%%%%%%%% Layout
\newcommand{\abstand}{\dimexpr\paperwidth/21\relax}
\setbeamersize{text margin left=\abstand,text margin right=\abstand}
\setlength{\columnsep}{\abstand}
\setlength{\unitlength}{1in}% beamer: 5" x 3.75" (4:3)

%%%%%%%%%%%%%% Fonts 
\setmainfont{Arial}
\setsansfont{Arial}
\setmonofont{Fira Mono}%Fira Mono, Ricty Diminished Discord
\renewcommand\UrlFont{\rmfamily} % passt URL an


%%%%%%%%%%%%%% Titlepage
\setbeamerfont{title}{size=\LARGE}
\setbeamerfont{subtitle}{size=\Large}
\setbeamerfont{author}{size=\small}
\setbeamerfont{date}{size=\small}
\setbeamerfont{institute}{size=\scriptsize}

\defbeamertemplate*{title page}{customized}[1][]
{\centering\vspace{\baselineskip}
    \color{FUblue}\usebeamerfont{title}\inserttitle\par
    \usebeamerfont{subtitle}%
    \vspace{0.5\baselineskip}
    \usebeamercolor[fg]{subtitle}\insertsubtitle\par
    \vspace{1\baselineskip}
    \usebeamerfont{author}\insertauthor\par
    \vspace{1.5\baselineskip}
    \textcolor{DBblue}{\rule{\textwidth}{1pt}}\par
    \vspace{2\baselineskip}
    \usebeamerfont{institute}\insertinstitute\par
    \vspace{1.5\baselineskip}
    \small\usebeamerfont{date}\insertdate\par
  %\usebeamercolor[fg]{titlegraphic}\inserttitlegraphic
}

% \insertshortinstitute
% \insertshorttitle
% \insertshortauthor
% \insertshortdate
% \insertframenumber
% \inserttotalframenumber
% \thesection
% \insertsection

%%%%%%%%%%%%%% Headline 
\setbeamertemplate{headline}{%
%   \leavevmode% horizontal mode is entered
%   \begin{beamercolorbox}[wd=\paperwidth,
% 	ht=2ex, dp=1ex, leftskip=1pt,left]%
%     {section in head/foot}
%     \insertsectionnavigationhorizontal{\paperwidth}{}{\hskip0pt plus1filll}
%   \end{beamercolorbox}%
%   \vskip0pt
%   \begin{beamercolorbox}[wd=\paperwidth,
%     ht=4ex, dp=1.125ex]%
%     {subsection in head/foot}
%     \insertsubsectionnavigationhorizontal%
%       {\paperwidth}{}{\hskip0pt plus1filll}
%   \end{beamercolorbox}
}


%%%%%%%%%%%%%% frametitle
\setbeamerfont{frametitle}{size=\large}
\setbeamertemplate{frametitle}{%
  \vskip-2pt
  \begin{beamercolorbox}[wd=\paperwidth, ht=2.5ex, dp=1.25ex, left]{frametitle}%
    \hskip\abstand%
    \insertframetitle
  \end{beamercolorbox}%
  \vskip-1pt%
  \begin{beamercolorbox}[wd=\paperwidth,
    ht=1.2ex, dp=0.55ex, left]%
    {subsection in head/foot}%
    \tiny%
    \insertsectionnavigationhorizontal%
      {\paperwidth}{\hskip\dimexpr0.5\abstand\relax}%
      {\hskip0pt plus1filll}% glue stretching
  \end{beamercolorbox}
}

%%%%%%%%%%%%%% Footline 
\newcommand{\footlinetext}{
\insertshortauthor\vskip1.5pt
\insertshortdate}

\setbeamerfont{footline}{size=\tiny,
series=\normalfont}

\setbeamertemplate{footline}{
% erzeugt den Fortschrittsbalken
\color{DBblue!60}%!60
\rule{\dimexpr\textwidth*\insertframenumber/\inserttotalframenumber\relax}{2pt}% horizontal line
\color{FUgreen!60}%
\rule{\dimexpr\textwidth-\textwidth*\insertframenumber/\inserttotalframenumber\relax}{2pt}
\vskip0pt%
% fügt Daten hinzu
\leavevmode% horizontal mode is entered
\hspace{\abstand}%
\begin{beamercolorbox}% Logo
  [wd=\abstand, ht=\dimexpr\abstand+1.5pt\relax, dp=1.5pt, left]
  {whitebox}%
  \includegraphics[height=\abstand]{img/agdb-logo}
\end{beamercolorbox}%
\hspace{0.5\abstand}%
\begin{beamercolorbox}% Titel
  [wd={\dimexpr\paperwidth-10\abstand\relax},
  ht=\dimexpr\abstand+1.5pt\relax, dp=1.5pt, left]% leftskip=2ex
  {whitebox}%
  \raisebox{6.75pt}{%
  \begin{minipage}{\linewidth}
  \footlinetext
  \end{minipage}}
\end{beamercolorbox}%
\hfill%
\begin{beamercolorbox}% Logo
  [wd=5\abstand, ht=\dimexpr\abstand+1.5pt\relax, dp=1.5pt, right]
  {whitebox}%
  \includegraphics[height=\abstand]
  {img/FULogo_RGB}\hspace{0.5\abstand}
\end{beamercolorbox}%
\ifnum\value{framenumber}<1%
  \begin{beamercolorbox}% Seitennummer
    [wd=\abstand, ht=\dimexpr\abstand+1.5pt\relax,
    dp=1.5pt, center]
    {whitebox}%
  \end{beamercolorbox}%
\else%
  \begin{beamercolorbox}% Seitennummer
    [wd=\abstand, ht=\dimexpr\abstand+1.5pt\relax,
    dp=1.5pt, center]
    {framenumber}%
    \raisebox{2.2ex}{\textnormal{%
    \insertframenumber{}%~/~\inserttotalframenumber
    }}
  \end{beamercolorbox}%
\fi%
}


%%%%%%%%%%%%%% TOC 
\setbeamertemplate{section in toc}[sections numbered]
\setbeamertemplate{subsection in toc}[subsections numbered]
\setbeamertemplate{section in toc}{ \textbf{\textcolor{DBblue}{\inserttocsectionnumber}}~\inserttocsection\par}
\setbeamertemplate{subsection in toc}{\hspace{0.6em}\color{FUblue!80}\inserttocsectionnumber.\inserttocsubsectionnumber~\inserttocsubsection\par}

\setcounter{tocdepth}{1} % TOC enthält section bis subsection


%%%%%%%%%%%%%% Enumitem 
\setlist[enumerate]{font=\color{DBblue}\bfseries, leftmargin=*}
\setlist[enumerate,1]{label=\arabic*, ref=\arabic*}
\setlist[enumerate,2]{label*=.\arabic*, ref=\theenumi.\arabic*}
\setlist[enumerate,3]{label*=.\arabic*, ref=\theenumii.\arabic*}
\setlist[itemize]{font=\color{DBblue}\bfseries, leftmargin=*}
\setlist[itemize,1]{label=%
	$\blacktriangleright$, ref=\labelitemi}
\setlist[itemize,2]{label=\small%
	$\blacktriangleright$, ref=\labelitemii} % \bullet
\setlist[itemize,3]{label=\footnotesize%
	$\blacktriangleright$, 
    ref=\labelitemiii}%textbullet, diamond
\setlist[description]{font=\color{DBblue}\bfseries, leftmargin=*}


%%%%%%%%%% Referenzen %%%%%%%%%%%
\usepackage[style=numeric, bibencoding=utf8, backend=biber, sorting=none, maxbibnames=99]{biblatex}% für bibliographie, style=authortitle/numeric/ieee, backend=biber/bibtex/bibtex8, bibencoding=ascii/utf8
\setbeamertemplate{bibliography item}{\insertbiblabel} % setzt Nummerierung im Lit.Verzeichnis
\usepackage{csquotes} % notwendig, wenn man babel und bibtex benutzt
\usepackage{silence}% Filter warnings issued by package biblatex starting with "Patching footnotes failed"
\WarningFilter{biblatex}{Patching footnotes failed}
\addbibresource{src/bibliography.bib}
% redefines Styles
\DeclareFieldFormat{isbn}{}

\usepackage{xpatch}
\xpatchbibmacro{author}{\printnames{author}}{\mkbibbold{\printnames{author}}}{}{} % author bold


%%%%%%%%%%%%% Farben %%%%%%%%%%%%%
\let\definecolor=\xdefinecolor
\definecolor{FUgreen}{RGB}{153,204,0}
\definecolor{FUblue}{RGB}{0,51,102}
\definecolor{DBblue}{RGB}{0,102,204}
\definecolor{DBbluedark}{RGB}{0,51,102}

\setbeamercolor{palette primary}{bg=FUblue,fg=white}
\setbeamercolor{palette secondary}{bg=FUblue,fg=white}
\setbeamercolor{palette tertiary}{bg=FUblue,fg=white}
\setbeamercolor{palette quaternary}{bg=FUblue,fg=white}
\setbeamercolor{title}{fg=DBbluedark, bg=white}
\setbeamercolor{subtitle}{fg=DBbluedark, bg=white}
\setbeamercolor{titlelike}{parent=structure}
\setbeamercolor{author}{fg=DBbluedark, bg=white}
\setbeamercolor{frametitle}{fg=white,bg=FUblue}
\setbeamercolor{section in head/foot}{fg=DBbluedark,bg=white}
\setbeamertemplate{section in head/foot shaded}[default][50]
\setbeamercolor{subsection in head/foot}{fg=DBbluedark,bg=lightgray}
\setbeamercolor{framenumber}{fg=white, bg=FUblue}
\setbeamercolor{whitebox}{fg=DBbluedark,bg=white}
% Standard block
\setbeamercolor{block title}{fg=white, bg=FUblue}
\setbeamercolor{block body}{fg=DBbluedark, bg=lightgray!30}
% Alert block
\setbeamercolor{block title alerted}{fg=white, bg=red}
\setbeamercolor{block body alerted}{fg=DBbluedark, bg=red!20}
% Example block
\setbeamercolor{block title example}{fg=white, bg=FUblue}
\setbeamercolor{block body example}{fg=DBbluedark, bg=lightgray!30}
\setbeamercolor{itemize item}{fg=DBblue} % all frames will have red bullets
\setbeamercolor{enumerate item}{fg=DBblue}

\hypersetup{
    colorlinks,
    linktocpage,
    allcolors=FUblue,
    urlcolor=DBblue,
    %linkcolor=red, % färbt auch Menü und Shorttitle
    breaklinks
}
% setzt die globale Schriftfarbe auf DBbluedark
\makeatletter
\newcommand{\globalcolor}[1]{%
  \color{#1}\global\let\default@color\current@color
}
\makeatother
\AtBeginDocument{\globalcolor{DBbluedark}}

% Bibliography entries:
\setbeamercolor{bibliography entry author}{fg=FUblue}
\setbeamercolor{bibliography entry title}{fg=FUblue} 
\setbeamercolor{bibliography entry location}{fg=FUblue} 
\setbeamercolor{bibliography entry note}{fg=FUblue}
\setbeamercolor{bibliography item}{fg=DBblue}

%%%%%%%%%%%% Blocks %%%%%%%%%%%%%
% passt die Größe der Blöcke an \linewidth an an zieht den Rand ab, den die Blöcke mit sich bringen
\addtobeamertemplate{block begin}{\flushright\setlength{\textwidth}{\linewidth-0.5\abstand}}{}
\addtobeamertemplate{block example begin}{\flushright\setlength{\textwidth}{\linewidth-0.5\abstand}}{}
\addtobeamertemplate{block alerted begin}{\flushright\setlength{\textwidth}{\linewidth-0.5\abstand}}{}

%%%%%%%%%%%% Umgebung %%%%%%%%%%%%
\newenvironment{frameWithPicture}[2]{% BEGIN
  \begingroup%
  \setbeamertemplate{background}{%
    \begin{picture}(5,3.75)(0,-0.27)
      \includegraphics[width = \paperwidth,
      height=\dimexpr\paperheight-5.5ex-6pt-\abstand\relax,
      %keepaspectratio
      ]{#2}
    \end{picture}%
  }%
  \begin{frame}{#1}
}
{% END
  \end{frame}%
  \endgroup%
}

%%%%%%%%%%%%%%%%%% Listings %%%%%%%%%%%%%%%%%%%
\renewcommand{\lstlistingname}{Source Code}% Listing -> Source Code
\lstloadlanguages{Python, R, HTML, Haskell, Java, SQL} 
\lstset{
   basicstyle=\color{DBbluedark}\small\ttfamily\selectfont,	% \scriptsize the size of the fonts that are used for the code
   backgroundcolor = \color{gray!20},	% legt Farbe der Box fest
   breakatwhitespace=false,	% sets if automatic breaks should only happen at whitespace
   breaklines=true,			% sets automatic line breaking
   captionpos=b,				% sets the caption-position to bottom, t for top
   commentstyle=\color{gray}\selectfont,% comment style
   frame=single,				% adds a frame around the code
   keepspaces=true,			% keeps spaces in text, useful for keeping indentation
							% of code (possibly needs columns=flexible)
   keywordstyle=\bfseries\color{DBblue}\selectfont,% keyword style
   ndkeywordstyle=\color{DBbluedark}\bfseries\selectfont,
   numbers=left,				% where to put the line-numbers;
   							% possible values are (none, left, right)
   numberstyle=\scriptsize\color{FUblue}\selectfont,	% the style that is used for the line-numbers
   numbersep=9pt,			% how far the line-numbers are from the code
   stepnumber=1,				% nummeriert nur jede i-te Zeile
   showspaces=false,			% show spaces everywhere adding particular underscores;
							% it overrides 'showstringspaces'
   %showstringspaces=false,	% underline spaces within strings only
   showtabs=false,			% show tabs within strings adding particular underscores
   flexiblecolumns=false,
   %tabsize=1,				% the step between two line-numbers. If 1: each line will be numbered
   stringstyle=\color{red}\ttfamily\selectfont,	% string literal style
   numberblanklines=false,				% leere Zeilen werden nicht mitnummeriert
   xleftmargin=\dimexpr\abstand+2pt\relax,	% Abstand zum linken Layoutrand
   framexleftmargin=\dimexpr\abstand-2pt\relax,
   xrightmargin=\dimexpr\abstand+2pt\relax,					% Abstand zum rechten Layoutrand
   framexrightmargin=\dimexpr\abstand-2pt\relax,
   aboveskip=2ex, 
}

\lstdefinestyle{html}{
   language=HTML,
}
\lstdefinestyle{hs}{
   language=Haskell,
}
\lstdefinestyle{sql}{
   language=SQL,
}
\lstdefinestyle{pseudo}{
   language = Python,
   mathescape = true,
   keywords = {do, procedure, end, while, if, else},
}
\lstdefinestyle{java}{
	language=Java,
    keywords={typeof, new, true, false, catch, function, return, null, catch, switch, var, if, in, while, do, else, case, break},
    ndkeywords={class, export, boolean, throw, implements, import, this},
	extendedchars=true,% lets you use non-ASCII characters;
    % for 8-bits encodings only, does not work with UTF-8
}
\lstdefinestyle{R}{
	language=R,
	extendedchars=true,% for 8-bits encodings only,
    % does not work with UTF-8
}